\documentclass{article}
\usepackage{amssymb}
\usepackage{amsmath}
\title{\textbf{ECE 6560 Final Project - Image Smoothing}}
\author{Alexander Domagala}
\date{Spring 2024}

\begin{document}
  \maketitle

  \section{Problem Description}
  Although image processing techniques began to appear around the 1960s, the proliferation of inexpensive
  digital computing power has greatly widened the application pool. Image processing techniques
  can now be found in areas such as digital photography, medical imaging, and object detection/tracking.
  \\

  \noindent
  All these application areas can be affected by a very common problem: image noise.
  Noise can render further processing ineffective, thus there exist a variety of approaches by which
  we can attempt to smooth and denoise our images. Traditional image processing techniques
  may rely on fixed kernels that can be swept over an image in order to smooth it. However,
  this type of approach will result in uniform smoothing can blur the edges of an image.
  \\

  \noindent
  We can instead leverage PDEs to describe how an image should be updated depending on its local characteristics.
  More specifically, we will examine Ansiotropic Diffusion. This technique can be used to reduce image noise
  while lessening the blurring that is done to edges. This project will attempt to develop PDEs that
  can be used to reduce noise in high/low contrast images.
  \\

  \noindent
  We need more information about the actual high/low contrast problem!
  \\



  \newpage
  \section{Mathematical Modeling}
  Before explicitly developing our PDEs, we must first understand what behavior we want our system to have.
  The calculus of variations can be used to minimize an energy functional. The choice of setup for the
  energy functional will determine the system behavior.
  \\

  \noindent
  Let's begin by defining our image as
    \begin{center}
      $I(x,y)$
    \end{center}
  Note that the derivations in sections (2) and (3) are being performed
  exclusively in continuous space: $x,y \in \mathbb{R}$.
  \\
  
  \newpage
  \noindent










  \newpage
  \section{Derivation of PDE}

    \noindent Now, let's introduce the Euler-Lagrange equation
      \begin{center}
        \begin{tabular}{l}
          $L_{f} - \frac{\partial}{\partial x}L_{f'} = 0$ (1-D)\\
          $L_{I} - \frac{\partial}{\partial x}L_{I_{x}} - \frac{\partial}{\partial y}L_{I_{y}} = 0$ (2-D)\\
        \end{tabular}
      \end{center}
    \vspace{12pt}

    \noindent
    We can begin working towards obtaining our PDE by setting up a gradient descent
      \begin{center}
        \begin{tabular}{l}
          $I_{t} = -\nabla_{I}E$\\
          $I_{t} = -L_{I} + \frac{\partial}{\partial x}L_{I_{x}} + \frac{\partial}{\partial y}L_{I_{y}}$
        \end{tabular}
      \end{center}
    \vspace{12pt}

    \noindent
    We will now have to compute terms $L_{I}$, $L_{I_{x}}$, $L_{I_{y}}$ using the previously obtained energy functional
      \begin{center}
        \begin{tabular}{l}
          \vspace{12pt}
          $L(I,I_{x},I_{y},x,y) = \frac{\lambda}{1+e^{\alpha}}$, where $\alpha = -\frac{1}{\beta}(\| \nabla_{I} \|)$\\
          $L_{I} = \frac{\partial}{\partial I}(L)$\\
          $L_{I} = 0$\\
          $L_{I_{x}} = \frac{\partial}{\partial I_{x}}(L)$\\
          $L_{I_{x}} = \frac{\lambda}{\beta} \frac{e^\alpha}{(1+e^{\alpha})^2} \frac{I_{x}}{\sqrt{I_{x}^2 + I_{y}^2}}$\\
          $L_{I_{y}} = \frac{\partial}{\partial I_{y}}(L)$\\
          $L_{I_{y}} = \frac{\lambda}{\beta} \frac{e^\alpha}{(1+e^{\alpha})^2} \frac{I_{y}}{\sqrt{I_{x}^2 + I_{y}^2}}$\\
        \end{tabular}
      \end{center}
      \vspace{12pt}

    \noindent
    Now that we have obtained our expressions for $L_{I_{x}}$ and $L_{I_{y}}$, we must compute their partial derivatives
    as shown by the Euler-Lagrange equation. This will be shown for $\frac{\partial}{\partial x}L_{I_{x}}$. $\frac{\partial}{\partial y}L_{I_{y}}$ will be obtained by examining
    the expression of $\frac{\partial}{\partial x}L_{I_{x}}$.\\

    \noindent
    Let $\phi$ denote $\frac{e^\alpha}{(1+e^{\alpha})^2}$ and let $\gamma$ denote $\frac{I_{x}}{\sqrt{I_{x}^2 + I_{y}^2}}$.
    We can begin finding $\frac{\partial}{\partial x}L_{I_{x}}$ by using the product-rule $\frac{\partial}{\partial x}(\gamma)\phi + \frac{\partial}{\partial x}(\phi)\gamma$.
    We will start with the left side of the sum. Note that we must include $\frac{\lambda}{\beta}$ in the final expression.\\
    \begin{center}
      \begin{tabular}{l}
        $\frac{\partial}{\partial x}(\gamma)\phi$\\
        $\frac{\partial}{\partial x}(\frac{I_{x}}{\sqrt{I_{x}^2 + I_{y}^2}})\phi$\\
        $(\frac{I_{xx}}{(I_{x}^2 + I_{y}^2)^\frac{1}{2}} + \frac{I_{x}}{(I_{x}^2 + I_{y}^2)^\frac{3}{2}}(I_{x}I_{xx} + I_{y}I_{xy}))\phi$
      \end{tabular}
    \end{center}
    \vspace{12pt}
    









    \newpage
    \noindent
    We can now examine the right side of $\frac{\partial}{\partial x}(\gamma)\phi + \frac{\partial}{\partial x}(\phi)\gamma$.
    \begin{center}
      \begin{tabular}{l}
        $\frac{\partial}{\partial x}(\phi)\gamma$\\
        $\frac{\partial}{\partial x}(\frac{e^\alpha}{(1+e^{\alpha})^2})\gamma$\\
        $\frac{\partial}{\partial x}((e^\alpha)(1+e^{\alpha})^{-2})\gamma$\\
      \end{tabular}
    \end{center}

    \noindent
    We see that we will need to again perform the product-rule between $(e^\alpha)$ and\\
    $(1+e^{\alpha})^{-2}$. Taking the partial derivative of $(e^\alpha)$
    \begin{center}
      \begin{tabular}{l}
        $-\frac{1}{\beta} e^{\alpha} \frac{1}{(I_{x}^2 + I_{y}^2)^\frac{1}{2}} (I_{x}I_{xx}+I_{y}I_{xy})$
      \end{tabular}
    \end{center}

    \noindent
    Taking the partial derivative of $(1+e^{\alpha})^{-2}$
    \begin{center}
      \begin{tabular}{l}
        $-2(1+e^{\alpha})^{-3} (-\frac{1}{\beta} e^{\alpha} \frac{1}{(I_{x}^2 + I_{y}^2)^\frac{1}{2}} (I_{x}I_{xx}+I_{y}I_{xy})) $
      \end{tabular}
    \end{center}

    \noindent
    Thus, after factoring common terms, $\frac{\partial}{\partial x}((e^\alpha)(1+e^{\alpha})^{-2})$ yields
    \begin{center}
      \begin{tabular}{l}
        $[-\frac{1}{\beta} (e^\alpha) (\frac{1}{(I_{x}^2 + I_{y}^2)^\frac{1}{2}}) (I_{x}I_{xx}+I_{y}I_{xy})] [(1+e^{\alpha})^{-2} + (e^\alpha)(-2(1+e^{\alpha})^{-3})]$\\
      \end{tabular}
    \end{center}
    \vspace{12pt}
    \vspace{12pt}

  \noindent
  We have reached the final expression for $\frac{\partial}{\partial x}L_{I_{x}}$
  \begin{center}
    \begin{tabular}{l}
      \vspace{12pt}
      $\frac{\partial}{\partial x}L_{I_{x}} =$\\
      \vspace{12pt}
      $\frac{\lambda}{\beta}[ (\frac{I_{xx}}{(I_{x}^2 + I_{y}^2)^\frac{1}{2}}) - (\frac{I_{x}}{(I_{x}^2 + I_{y}^2)^\frac{3}{2}}) (I_{x}I_{xx} + I_{y}I_{xy}) (\frac{e^\alpha}{(1+e^{\alpha})^2}) +$\\
      \vspace{12pt}
      $(\frac{I_{x}}{(I_{x}^2 + I_{y}^2)^\frac{1}{2}}) [-\frac{1}{\beta} (e^\alpha) (\frac{1}{(I_{x}^2 + I_{y}^2)^\frac{1}{2}}) (I_{x}I_{xx}+I_{y}I_{xy})] [(1+e^{\alpha})^{-2} + (e^\alpha)(-2(1+e^{\alpha})^{-3})]]$
    \end{tabular}
  \end{center}
  \vspace{12pt}

  \noindent
    $\frac{\partial}{\partial y}L_{I_{y}}$ can be obtained by modifying $\frac{\partial}{\partial x}L_{I_{x}}$
    \begin{center}
      \begin{tabular}{l}
        \vspace{12pt}
        $\frac{\partial}{\partial y}L_{I_{y}} =$\\
        \vspace{12pt}
        $\frac{\lambda}{\beta}[ (\frac{I_{yy}}{(I_{x}^2 + I_{y}^2)^\frac{1}{2}}) - (\frac{I_{y}}{(I_{x}^2 + I_{y}^2)^\frac{3}{2}}) (I_{x}I_{xy} + I_{y}I_{yy}) (\frac{e^\alpha}{(1+e^{\alpha})^2}) +$\\
        \vspace{12pt}
        $(\frac{I_{y}}{(I_{x}^2 + I_{y}^2)^\frac{1}{2}}) [-\frac{1}{\beta} (e^\alpha) (\frac{1}{(I_{x}^2 + I_{y}^2)^\frac{1}{2}}) (I_{x}I_{xy}+I_{y}I_{yy})] [(1+e^{\alpha})^{-2} + (e^\alpha)(-2(1+e^{\alpha})^{-3})]]$
      \end{tabular}
    \end{center}
    \vspace{12pt}





    \newpage
    \noindent
    Our final gradient-descent PDE is
    \begin{center}
      \begin{tabular}{l}
        \vspace{12pt}
        $I_{t} = \frac{\lambda}{\beta}[ (\frac{I_{xx}}{(I_{x}^2 + I_{y}^2)^\frac{1}{2}}) - (\frac{I_{x}}{(I_{x}^2 + I_{y}^2)^\frac{3}{2}}) (I_{x}I_{xx} + I_{y}I_{xy}) (\frac{e^\alpha}{(1+e^{\alpha})^2}) +$\\
        \vspace{12pt}
        $(\frac{I_{x}}{(I_{x}^2 + I_{y}^2)^\frac{1}{2}}) [-\frac{1}{\beta} (e^\alpha) (\frac{1}{(I_{x}^2 + I_{y}^2)^\frac{1}{2}}) (I_{x}I_{xx}+I_{y}I_{xy})] [(1+e^{\alpha})^{-2} + (e^\alpha)(-2(1+e^{\alpha})^{-3})]+ $\\
        \vspace{12pt}
        $(\frac{I_{yy}}{(I_{x}^2 + I_{y}^2)^\frac{1}{2}}) - (\frac{I_{y}}{(I_{x}^2 + I_{y}^2)^\frac{3}{2}}) (I_{x}I_{xy} + I_{y}I_{yy}) (\frac{e^\alpha}{(1+e^{\alpha})^2}) +$\\
        \vspace{12pt}
        $(\frac{I_{y}}{(I_{x}^2 + I_{y}^2)^\frac{1}{2}}) [-\frac{1}{\beta} (e^\alpha) (\frac{1}{(I_{x}^2 + I_{y}^2)^\frac{1}{2}}) (I_{x}I_{xy}+I_{y}I_{yy})] [(1+e^{\alpha})^{-2} + (e^\alpha)(-2(1+e^{\alpha})^{-3})]]$
        \vspace{12pt}
      \end{tabular}
    \end{center}

    \noindent
    Where $\alpha = -\frac{1}{\beta}(\| \nabla_{I} \|)$\\




































  \newpage
  \section{Discretization and Implementation}


  \section{Experimental Results}


  \section{Summary}

\end{document}