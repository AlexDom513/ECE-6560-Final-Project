\documentclass{article}
\title{\textbf{ECE 6560 Final Project - Image Smoothing}}
\author{Alexander Domagala}
\date{Spring 2024}

\begin{document}
  \maketitle

  \section{Problem Description}
  First describe the problem you wish to solve (or attempt to solve) without
  using mathematical language (or at least as little as possible), so that
  somebody with an engineering background but who doesn't know the
  specific mathematical tools discussed in this course will still clearly
  be able to follow the description.

  \section{Mathematical Modeling}
  Next, "translate", your problem into a mathematical problem. For example,
  if you are using variational gradient descent PDE's, you may want to explain
  how to formulate an energy functional that captures the aspects of your problem
  into a mathematical expression.  You don't necessarily need to introduce the
  PDE yet in this section. This section, should just focus on how to encode
  your problem mathematically.

  \section{Derivation of PDE}
  Next, show how the PDE you intend to use to solve your problem is derived
  or formulated from your mathematical model. For example, in the case you are
  using a variational gradient descent PDE, here is where you will show how to
  obtain your gradient flow PDE from the energy function that you already
  presented. Make sure it is clear to me that you understand how the PDE is
  obtained (don't just present it with no explanation).

  \section{Discretization and Implementation}
  Now, show how the PDE you are using should be discretized and implemented
  on the computer.  You should justify the choices you make in the discretization
  (why, for example, are you using central differences or upwind difference or
  entropy differences for the spatial derivatives, how and why are you choosing
  the time step if your PDE is a time evolution PDE). Make sure it is clear
  to me that you understand the reasons behind your choices (don't just present
  them without explanation).

  \section{Experimental Results}
  Now present some experimental results from applying your discretized
  PDE methods on data. The data can be synthetically generated or real data
  (real data is not important for this project). If you are exploring more
  than one PDE model (and/or testing the effect of changing some parameters in
  your PDE model), then make sure you present multiple experimental results.
  Attempt to extract/present some quantitative information to compare results
  rather than just showing only images. For example, if you are doing anisotropic
  image diffusing to preserve edges, you can devise any sort of reasonable
  quantitative measure to compare the edge preservation effects between
  different PDE choices, and then present a table of the comparisons (or a plot
  showing the way this value changes with the number of PDE iterations for
  each choice of your PDE).

  \section{Summary}
  Finally, discuss what we have learned (beyond what was already covered in
  lecture) as a result of your project. This can be both positive and negative.
  In otherwords, you may have discovered a weakness as well as a strength in
  your PDE based approach while doing your experiments. Try to related the
  strengths and/or weakness discovered during your experimental testing to
  the assumptions (or lack of assumptions) made in the mathematical formulation
  of your problem. In the case of weakness, suggest what you might do in
  the future if you had more time to reformulate, adjust, or improve your
  model and/or your discrete implementation algorithm.

\end{document}